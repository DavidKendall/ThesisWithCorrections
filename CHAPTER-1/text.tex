\chapter{Introduction and overview}\label{introduction}
%\section{Swarming}
\emph{Swarming} in the animal kingdom of ants, bees, fish and birds for instance has long been studied by scientists. From these studies mathematical models and algorithms have evolved. The models and algorithms have in turn captured the interest of computing scientists who are interested in applying them to large groups of autonomous mobile \emph{agents} (`robots'). The cooperative coordination of these agents can take many forms such as following a set path~\cite{HCS:09}, existing in a static space~\cite{EP:10, GP:02, GP:04} or foraging as a colony~\cite{HER:11, GK:07}. One of the attributes of swarms that has captured the interest of scientists is that the models and algorithms used to coordinate them are generally sets of simple rules. These simple rules cause the agents to appear to work cooperatively. Swarms can also exhibit features or behaviours which were not expected. This is due to the global effects of the simple algorithms being executed in a distributed manner. These unexpected behaviours are known as \emph{`emergent behaviours'}~\cite{RM:11, RMT:15}.

The ability to have autonomous agents working collaboratively has led to the development of systems that use this phenomenon to solve problems in different ways. In 1986, before swarming was being widely used as a technology, there was an explosion at the Chernobyl nuclear power station. To determine the extent of the destruction, a robot was deployed to inspect the reactor base and carry out surveillance of the damage to the building~\cite{JA:98}. The robot was manually operated and had no autonomous capabilities. More recently, in 2015, a project was undertaken to carry out surveillance of forest fire perimeters~\cite{BSB:15}. The difference between these two surveillance projects was that rather than employing a single robot to carry out the surveillance of the forests, a swarm of \emph{decentralised} autonomous agents was deployed. This illustrates a developing trend of applying swarms to the problem of environmental surveillance. 

\section{Biological swarms}
Swarming has been identified in many species including fish, birds, insects, and more recently, mammals~\cite{SFDCIC:15}. It is believed that this behaviour has evolved over thousands of years, through natural selection, as a mechanism to improve the survival of species~\cite{CZ:07}.

Fish swarm in the form of shoals~\cite{PMRT:14} in an attempt to make it more difficult for predators to catch them. It is thought that grouping together makes it difficult for a predator to isolate an individual~\cite{MP:87}.

Birds flock together for the same reason as fish, to increase their survival prospects, but also to improve the efficiency of area coverage when feeding~\cite{PMRT:14}. In the case of starlings and their evening murmurings~\cite{YSCGL:12}, it is believed that the flock is identifying an optimal roost for itself, while ensuring its survival by disorientating predators. The disorientation for the predator stems from the distribution of the individuals in the flock as it moves~\cite{CCGP:10}.

Locusts swarm when feeding to make best use of the food resource by increasing the coverage of an area to ensure the resources are exhausted~\cite{GSSC:12, TDEB:12}.

Ants and bees live in colonies~\cite{MK:15, SVP:06} and it is believed they swarm to make best use of their resources and also to allow specialisations within their communities. The specialised individuals would not be able to survive alone but, as part of a colony, they add value to the group. Ants for instance have specialisations such as soldier ants for defence and nursing ants to look after pupae. In bee swarms there are workers that forage and queens and drones that remain in the hive~\cite{BBKA:ND}.

More recently there has been research to show that swarming-based behaviours exist in higher order animals such as baboons~\cite{SFDCIC:15}, where they use a consensus-based decision process to determine a troupe's movement. Yao and Hwang have analysed human behaviour and found humans exhibit \emph{boid-based} behaviours when in groups through \emph{cohesion}~(\autoref{sec:Cohesion1}) and \emph{repulsion}~(\autoref{sec:Repulsion1}) which they refer to as \emph{separation}, the third component they discuss is \emph{alignment}~(\autoref{sec:Direction1}), which is a consensus-based directional movement~\cite{YH:14}. Reynolds~(ref{REY:87}) describes this same structure when describing \emph{boid-based} movement.

All these adaptations and behaviours have led to the ecology community focusing on how these behaviours emerge and to use computer simulation to emulate the behaviour, and therefore understand the mechanisms the swarms use~\cite{ENNT:10}. In the case of analysing baboons, they used high-accuracy GPS trackers and with humans they used phone based GPS data~\cite{SFDCIC:15}.

The general consensus is that nature, through natural selection, refines behaviours to sustain a population or to help it adapt or expand. This has led to research into the mechanics of how animals interact to achieve these swarming effects~\cite{ENNT:10, RO:15, TDEB:12}. Passino has analysed bee populations in a hive~\cite{SVP:06} and authored books on bio-mimicry. He has also authored several papers on the computational theory of swarm stability with Gazi~\cite{GP:04, GP:02, GP:04a}. This shows a link between the natural world and computer science.

All these naturally occurring swarms provide paradigms that allow the categorisation of swarms. Naturally occurring swarms include foraging-based (bees), colony-based (ants) and flocking-based (fish/birds). These basic swarming models have been used to influence how computational models are designed to mimic the behaviours found in nature. These models can be applied to robotic swarms which are used for specific tasks based upon behavioural requirements~\cite{LG:14}.

\section{Computational swarms}
Computational swarms are inspired by aspects of biological swarms. The degree to which the biological swarm is emulated within the computational environment varies. A prominent feature that is frequently emulated is the cooperative behaviour of the swarm agents by simulating agents movements using repulsion and cohesion between the agents. The emergent behaviours that simple algorithms create through these agent interactions is the focus of this thesis.

\subsection{Foraging swarms}
Foraging swarms are composed of agents that emulate the natural world by carrying out tasks that involve a permanent base. The tasks are carried out by agents to ensure the colony survives or expands. The coordination in these types of swarm is for the colony to maintain itself by using scout agents to locate resources that are required and then to return those resources to the colony~\cite{GP:04a, HSWN:10, LWWA:07, LWWA:10}. The foraging component of this process is the locating of resources. There is also a communications component to foraging swarms. Foraging agents inform the rest of the colony of the resource locations to optimise the foraging tasks. Beeclust~\cite{PC:09, HER:11}, Swarm-agent~\cite{MFGAB:03} and other bee-inspired algorithms~\cite{LAA:13} are all implementations of this type of swarming behaviour.  

\subsection{Ant-colony swarms}
Ant-colony swarms~\cite{SWGC:04, MGFND:05, HSSC:10, MVZ:04} are similar to foraging swarms in terms of their interactions. The difference is in the way the agents communicate with each other. The agents move independently and there is no need for them to ensure proximity as the agents follow predefined routes. The agents are therefore independent in that there is no centralised coordinator and they act autonomously. The cooperation component of the swarm is realised by agents highlighting desirable or undesirable routes. The agents then follow the same trails back to a base and either reinforce or reduce the importance of the routes by adding or removing a pheromone~\cite{VBBLO:15}. The purpose of the trails are determined by identifying the needs of the colony centrally (at the base). In nature this is exactly how ant colonies function~\cite{JRD:03}. Some robotic ant-colony simulations include the concept of the `pheromone decay' process as found in biological ant colonies~\cite{RGJHR:08}. This process allows for changes in the priorities of the agents to be based on time as well as reinforcement as a colony propagates through, or exists within, an environment.

\subsection{Boid-based swarms}
Boid-based swarms, as originally defined by Reynolds~\cite{REY:87}, are modelled on the behaviour of fish and birds. They are composed of autonomous agents that are \emph{decentralised} and formulate their positions based upon an awareness of the location of their neighbours~\cite{JMM:08, CHRE:04, HSR:07}. The agents in a boid-based swarm are independent and each control their own position. The two major factors that create the swarming effect are \emph{cohesion} and \emph{repulsion}. Cohesion ensures the swarm has a tendency to stay together as a single entity. This has been used in the SmartBot project, where it has been found that cohesion promotes the collaboration of autonomous agents~\cite{DOR:09, DTGT:04, MFGAB:03}. Repulsion ensures that the agents do not collide, and when balanced with cohesion create a well-structured swarm. The balancing of these two factors is identified by Gazi and Passino in their swarm stability papers~\cite{GP:04, GP:02, GP:04a} and as part of the GUARDIAN project~\cite{SALGVPJ:08}. They also discuss cohesion and repulsion in their book on swarm optimisation~\cite{GP:11}.   

A directional component can also be incorporated into the movement of agents. The movement can be based on the direction of an agent's neighbours as in~\cite{REY:87, JMM:08}, where direction is referred to as alignment. Alignment is a consensus-based direction that the agents negotiate by communicating with each other. The negotiated direction is not based on a set goal that the swarm must move towards. 

Direction can also be applied as a goal as discussed by Hiroshi et al. and Navarro et al. in~\cite{HAY:08, IGMFM:08} where the direction is based on a position that the swarm must move towards and each agent is able to identify the direction locally. The goal can be determined based upon local environmental stimuli, such as temperature~\cite{PG:08}. If a swarm is to be used in an open air environment covering a large area, a GPS sensor can be used to determine its goal~\cite{SH:11}. 

\subsection{Centralised swarms}
The concept of centralised coordination is not seen in biological swarms.
Centralised swarms are inspired by the benefits of cooperative agents being used to solve a problem. The agents themselves are autonomous in terms of their function but their positional autonomy is removed and they are centrally managed. This centralised paradigm determines the type of tasks the swarm can be applied to~\cite{AZMD:15, MGFND:05, LADPC:07}. Centralised swarms are deployed into a known environment and a central controller coordinates the positional information~\cite{MP:09, I:01, NM:12, SB:93, MYP:09}. The model calculates positional requirements for all the agents and transfers that information to the agents through a communications infrastructure. This is different to decentralised swarms, such as boid-based swarms, that are predominantly based upon localised proximity field effects~\cite{BAF:06, BAFVM:06, BFV:07, BM:09}. Field effects are the omnidirectional ranges used by an autonomous agent to determine the proximity of nearby agents to determine their relationship~\cite{BAF:06}. Centralised swarms are different from swarms that use an internal communications infrastructure to negotiate roles and exchange information~\cite{OFM:07} such as the BEECLUST swarm~\cite{HER:11}. 

In a centralised swarm the positioning of the agents is entirely determined by a central controller, and communicated to them by it. The controller is a single point of failure and the communications overheads can be significant~\cite{NVC:15}. This adversely affects the reliability and scalability of the swarm.

The central processing of the algorithm is complex due to calculating multiple agent locations rather than a single location. The processing complexity can be overcome by increasing the performance of the central controller, but this will not overcome the communications problem. 
This type of swarm works well when creating predefined structures such as the tower building swarm~\cite{EG:11} or the knot tying quad-copters~\cite{AZMD:15} that have been developed as part of the research projects of D'Andrea in the department of Dynamic Systems and Control at ETH Zurich. The focus of this thesis  is on swarms that do not require a central controller. Control is distributed and each agent acts independently.

\section{Swarming applications}\label{sec:SwarmApplications} 
Many industries require the exploration or reconnaissance of environments that are not easily accessible by humans. Consider for instance disaster areas following earthquakes, or environments that are simply hard to survey due to their size, such as large scale commercial farms~\cite{BSB:15, HGCTREA:15}. 

There are occasions on which it is necessary to explore underground or enclosed spaces~\cite{IBE:08}. In mining, for example, the environment may be a labyrinth of tunnels that may be dangerous due to rock falls or toxic gas etc. Such environments may consist of many rejoining routes and dead ends. This type of work is best performed by swarms of autonomous robots~\cite{LG:14}.

An example of a large implementation of a swarming platform is Project Loon~\cite{KS:ND, G:13, RAV:13}. Google have completed trials and are now creating aerial platforms with high altitude balloons to provide communication infrastructures in remote areas of New Zealand~\cite{CBS:13, NZH:13}. There are also smaller scale projects investigating the use of swarms in surveying crops to check the health of plants~\cite{BSB:15}. This is to identify remedial actions that can improve crop yields. The forestry commission have carried out surveys of forest environments using swarms of UAVs (Unmanned Aerial Vehicles). All these applications require the agents not only to coordinate themselves within the swarm environment, but also to carry sensor arrays to detect environmental conditions.

There is a view that swarms can be made to interact with humans. In 2011, a trial took place that used swarms to assist individuals in the fire service~\cite{PAWN:11} as part of the GUARDIAN project~\cite{SALGVPJ:08}. In 2005 there was research by Stormont into the use of swarms to assist homeland first responders~\cite{STO:05}. The paper concluded that \emph{``the RoboCup goal of fully autonomous collaborative rescue robots by 2050 is a pretty good estimate''}.

It is clear that the application of swarms has increased and diversified into many industries. This has been made possible by the increased understanding of their capabilities. The work in this thesis further increases that understanding.

\section{Focus of the thesis}
This thesis takes its lead from swarming in the natural world and focuses on boid-based swarms with the addition of a directional component where necessary. The directional component will be applied as a global positional requirement of the swarm as used in large scale reconnaissance projects.

Although research into swarming algorithms can be carried out using both physical implementations~\cite{DOR:09, DTGT:04} and software simulations~\cite{BAF:06, GP:02}, the work covered in this thesis uses only software simulations. This makes possible the study of very large swarms over flexible time scales.

The application of swarms to solve large scale problems has increased as greater understanding of how swarms can be coordinated and monitored has improved. This thesis describes the development of a new metric for evaluating configurable coordination algorithms. This increases the understanding of how the dynamics of a swarm can be tailored to specific application areas. The algorithms, metric, and simulator have been developed as part of this thesis.

The thesis argues that the utility of a swarm in reconnaissance can be improved by exploiting emergent behaviours to improve the area coverage of goal based swarms when encountering obstacles. This could improve detection rates when swarms are used for searching  activities such as locating targets within a large area. These targets could be mountaineers in remote areas or livestock on commercial farms. 

The thesis also identifies behaviours that can be used to promote a \emph{self-healing} effect to improve the structure of a swarm~\cite{RS:08}. Self-healing is the ability of a swarm to remove `holes' from it's structure. This behaviour can also be applied to surrounding objects. The oil industry has been involved in several man-made disasters involving large scale oil spillages. Research into possible containment of these spillages has shown it is possible to use swarms to identify an oil slick's perimeter~\cite{ZFG:13}. This thesis shows that the self-healing effect can be applied to the task of containing an oil spillage.
 
\section{Contributions}
Navarro defined a set of metrics for the analysis of swarms~\cite{NIM:09}. These metrics were based on the positions of agents in a swarm and looked at average speed, density of the population, and variations in distances. This thesis proposes a new metric for swarm analysis. 

The new metric is based upon the inter-agent interactions and is independent of the distribution of the agents. The interactions are the magnitudes of the cohesion and repulsion vectors that the field effects and algorithms produce. These same vectors when summed and normalised produce the directional vector of each agent. By focusing on the interaction of the agents at the mathematical layer rather than just the spatial distribution the metric identifies the degree of influence each agent has upon its neighbours. The inter-agent interactions can be used as a comparative metric for different swarming algorithms. The new metric identifies the effects of different algorithms when they produce both regular and irregular spatial distributions. The metric can also be used to highlight specific states in a swarm such as when flood filling an area. i.e. the inter-agent magnitude increases without causing a spatial distance increase. This state identification can be used as an exit condition for an area filling task.

%% The new metric shows it is possible to analyse a swarm and show whether an algorithm is producing an efficient swarm movement and the degree to which the inter-agent repulsion and cohesion are influencing the swarm. The new metric will also allow the effects of algorithms to be monitored and compared. The metric will allow the identification of swarm `flooding' effect where agents are evenly dispersed throughout and area. The identification of the flooding is determined by the cohesion and repulsion fields being increased to a point where the increase has no effect on the inter-agent distances. 

This thesis introduces three directional algorithms that allow swarms to be applied to tasks such as search and rescue or reconnaissance. Most directional swarms use some kind of positioning system which all agents employ. This thesis demonstrates that it is possible to reduce the number of agents in a swarm employing a positioning system in a consistent manner such that the swarm still exhibits a directional bias. These new algorithms also reduce the gross energy consumption of the swarm making the swarm more energy efficient. The thesis also demonstrates that a reduction in the position system utilisation reduces the inter-agent disturbances.  

%% This thesis introduces two directional coordination algorithms and compares these to a standard goal-based algorithm. The standard goal-based algorithm is all agents in a swarm having a directional bias. The new methods reduce the number of agents in a swarm that require directional information to produce the directional effect, this also reduces the gross energy consumption when using directional sensors. 

This thesis demonstrates that emergent behaviours can be exploited to improve the structure of a swarm. Swarms, by consisting of many agents, are resistant to agent failures. However failures can occur and when they do they create gaps in the swarms structure where the failed agents were located. Swarms can also develop irregularly shaped perimeters with dents. Dents are concave deformations caused by deployment irregularities, external effects such as obstacles, or perimeter agents coming into contact with additional agents. These characteristics (anomalies) reduce the effectiveness of a swarm in some tasks due to the overall structure being non-uniform. This thesis addresses these specific issues by extending the basic swarming algorithm to produce a localised agent effect that has a global impact on the swarms structure by reducing and removing these anomalies thus `healing' the swarm.

%% This thesis will show how, by identifying certain swarm facets, it is possible to promote void reduction in a swarms structure and create a `healing effect'. Voids are areas in the swarm where there are breaks in the lattice formation. This technique will be applied in two ways. 
%% In a static swarm when an agent fails it causes an anomaly in the swarm's structure. This failure could be resource exhaustion, such as a battery running low, or a catastrophic event, such as a motor failure. This thesis addresses this issue by introducing an algorithm that produces a `void closing' effect. This effect makes the swarm coalesce into a coherent structure with no voids.

Riano~\cite{RM:11} has shown that there are hidden benefits in using swarms due to the emergent behaviours of group dynamics which can assist reconnaissance. Swarms are often modelled in environments that include obstacles that must be avoided~\cite{VG:05, BAB:12, TRI:15}. These obstacles can cause voids in a swarm. A void is an area within the body of a swarm where there are no agents. The void reduction technique developed in this thesis increase the ability of a swarm to reduce voids that are created by an obstacle when they are in the path of a goal based swarm. By using this `healing' effect to remove the voids it is possible to increasing the `coverage' around an obstacle. This builds upon the work by Geunho Lee and Nak Young Chong~\cite{GN:08}. 

Arkin et al., Fang et al., Krothapalli et al. and Luc Moreau state that agents in a swarm need to communicate with one another locally in order to maintain a swarm's structure~\cite{ABN:93, FAP:05, KD:99, MOR:05, PYP:01}. Hoff et al. have shown that localising communications to just neighbours is advantageous~\cite{HSWN:10} as it reduce the message propogation pathways within a swarm. Nithin et al. state that agents could have a central communications infrastructure that is independent of the agents~\cite{NVC:15}, as used in centralised swarms. 

Alternatively Higgins et al. and Navarro et al. state that inter-agent communications limit or impair a swarm's functionality~\cite{HTM:09, IGMFM:08}. This thesis proposes that a communications infrastructure is not required for the identification of features such as perimeters as local positional information is all that is required. Local positioning can be obtained without inter-agent communications by using sensors such as an omnidirectional camera.

This thesis demonstrates algorithms that are able to detect perimeters, which are the edges of a swarm, and perimeter anomalies (deformations) without the need for a global swarm based communications infrastructure. This removal of the need for message propagation allows the algorithms to be applied to arbitrary sized swarms.

%% This thesis proposes techniques to identify useful features such as perimeters and indentations without the need for agents to exchange data via a communications infrastructure.  The removal of the communications infrastructure allows the coordination algorithms to be applied to arbitrary size swarms. 

This thesis focuses on arbitrary sized swarms. Modelling large numbers of agents in a swarm is most practicably carried out using a simulator. The requirements of the swarm analysis using inter-agent interactions is a very specific requirement. Combining these two requirements a bespoke simulator is presented as part of this thesis. The simulator is designed using an object model approach with data capture and accurate modelling as the primary goals. The object model used in the simulator is similar to that described by Vankerkom and Yu~\cite{VY:04} and provides an extensible framework for the development of swarming applications. This thesis uses the framework to create two applications. A graphical scenario creation tool and a command line simulation tool.

\section{Structure of the thesis}
The rest of the thesis is structured as follows: Chapter 2 covers methods tools and techniques used to implement the coordination of agents in a swarming structure. Chapter 3 covers the simulator that has been developed in order to investigate the algorithms proposed as part of this thesis. Chapter 4 discusses the development and application of the metric that allows the analysis of the effect a particular swarming algorithm has on a swarm's internal movement. Chapter 5 presents the metric and shows how the metric can be used to identify the effects of algorithms and field effects on the structure of a swarm and how different inter-agent relationships can be identified. Chapter 6 discusses two methods of coordinating a goal-based swarm and a baseline for comparison. This chapter also identifies the changes these algorithms generate on the movements of agents within a swarm. Chapter 7 examines the emergent behaviours of void reduction on goal-based and stationary swarms. Chapter 8 discusses the use of field effects to create an area filling behaviour and demonstrates how the new metric can be used to identify an exit condition when the area filling is completed. Chapter 9 sums up all the findings of the thesis and identifies additional work that has been identified through the research carried out as part of this thesis.