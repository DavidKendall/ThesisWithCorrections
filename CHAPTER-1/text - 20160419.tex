\chapter{Introduction and Overview}\label{introduction}
\section{Swarming}
\emph{Swarming} is the collaboration of `things' such that they appear to exhibit some form or cooperative coordination~\cite{BS:13, BAFVM:06, BAF:06, BFV:07}. The `things' being agents which can be an animal, person or object that has a role or can produce an effect in an environment. The cooperative coordination of these agents can take many forms such as following a set path~\cite{HCS:09}, existing in a static space~\cite{EP:10, GP:02, GP:04} or foraging as a colony~\cite{HER:11, GK:07}. The main attribute of a swarm is that by using simple rules the agents appear to be working cooperatively. 
Swarms can also exhibit features or behaviours which were not expected, this is due to the global effects of simple algorithms being executed independently. These unexpected behaviours are known as emergent behaviours~\cite{RM:11, RMT:15}.
The ability to have autonomous agents working collaboratively has lead to the development of systems that use this phenomenon to solve problems in a different way. If surveillance of an area was required such as when the Chernobyl (1986) disaster occurred a single robot would have been used to explore the resultant aftermath~\cite{JA:98}. Following the research into swarming the benefits of distributing a task to multiple agents are being realised through projects such as swarm based forest surveys~\cite{BSB:15} (2015), and the tracking of forest fires~\cite{SOM:12} (2012).

\section{Biological swarms}
Swarming originates from nature and has been identified in many species including fish, birds, insects and more recently mammals~\cite{SFDCIC:15}. It is believed that this behaviour has evolved over thousands of years through natural selection as a mechanism to improve the probability of a species surviving~\cite{CZ:07}.

Fish swarm in the form of shoals~\cite{PMRT:14} in an attempt to make it more difficult for predators to catch them, it is believed that by grouping together a predator finds it difficult to isolate an individual.

Birds flock~\cite{PMRT:14} together for the same reason as fish, to increase there survival rate but also to improve the efficiency of area coverage when feeding. In the case of starlings and their evening murmurings~\cite{YSCGL:12} it is believed that the swarm is identifying an optimal roost for the flock while ensuring the flocks survival by distracting predators due to their numbers making it difficult for predators.

Locusts swarm when feeding to make best use of the food resource by increasing the coverage of an area to ensure the resources are exhausted~\cite{GSSC:12, TDEB:12} and when this occurs it is generally viewed as a plague.

Ants~\cite{MK:15} and Bees~\cite{SVP:06} live in colonies and it is believed they swarm to make best use of their resources and also to allow specialisations within their communities. The specialised individuals would not be able to surviving alone but as part of a colony they add value to the group and help it to survive. Ants for instance have specialisations such as soldier ants for defence and nursing ants to look after pupae. In Bee swarms there are drones that forage and a queen.

More recently there has been research to show swarming based behaviours exist in higher order animals such as Baboons~\cite{SFDCIC:15}.

All these adaptations and behaviours have lead to the ecology community to focus at how these behaviours emerge and are using computer simulation to try and emulate the behaviour and therefore understand the mechanisms the swarms utilise~\cite{ENNT:10}.

The general consensus is that nature through natural selection refines behaviours to sustain a population or to help it adapt or expand which has lead to research into understanding the mechanics behind how animals interact to achieve these swarming effects~\cite{ENNT:10, RO:15, TDEB:12}. One of the authors, Kevin M. Passino, in the article `Group decision making in honey bee swarms'~\cite{SVP:06} which is a paper analysing bee behaviour is also an author on several computation theory papers based on swarm stability~\cite{GP:04, GP:02, GP:04a} and also books on bio-mimicry and swarm stability. 

All these naturally occurring swarms can be catagorised based upon how the coordination is achieved. Some of these include Foraging (Bees), Colony based (Ants) and Boid based (Fish/Birds). These basic swarming models have been used to influence how a computational model is designed so as to mimic the behaviours found in nature, these models are then applied to robotic swarms which are then applied to solving a task~\cite{LG:14}.

\subsection{Foraging swarms}
Foraging swarms are composed of agents that emulate the natural world to carry out tasks that involve a permanent base and a selection of tasks that need to be carried out for the colony to survive or expand, this could be to collect food or building materials. The coordination in these types of swarms is to maintain the colony by using scout agents to locate resources that are required and then to return them to the colony when they are found~\cite{GP:04a, HSWN:10, LWWA:07, LWWA:10} and to inform the colony of the locations. These types of swarms, such as Beeclust~\cite{HER:11}, Swarm-agent~\cite{MFGAB:03} and other bee inspired algorithms~\cite{LAA:13} are all implementations of this type of swarming behaviour.  

\subsection{Ant colony swarms}
Ant colony swarms~\cite{SWGC:04, MGFND:05, HSSC:10, MVZ:04} are similar in construct to foraging swarms in terms of their interactions. The difference is in the way the agents communicate with each other. The agents move independently and there is no need for cohesion to ensure proximity as ant colonies follow predefined routes. The agents are therefore independent in that there is no centralised coordinator. The cooperation component of the swarm is realised by agents highlighting a desirable or undesirable routes. This is achieved by agents travelling to a location and creating a detectable trail then following the same trail back and either re enforcing it or reducing it by adding or removing a pheromone~\cite{VBBLO:15}. The nature of the trails are determined by identifying the needs of the colony which is negotiated at a central location. In nature this is exactly how ant colonies function~\cite{JRD:03}. Some robotic ant colony simulations include the concept of the pheromone decay process. This allows for changes in the priorities of the agents to be based on time as well as reinforcement as a colony propagates through, or exists within an environment.

\subsection{Boid based swarms}
Boid based swarms as originally defined by Reynolds~\cite{REY:87} are composed of autonomous agents that are decentralised and formulate their positions based upon an awareness of each other either as neighbours or as an entire swarm~\cite{JMM:08, CHRE:04, HSR:07}. The agents in a Boid based swarm are independent and each control their own positional algorithm. The two major factors that create the swarming effect are cohesion and repulsion. Cohesion ensures the swarm has a tendency to stay together as a single swarm, and repulsion ensures the agents do not collide. There is also a directional component that can be incorporated into the agents movements based on neighbour direction as in~\cite{REY:87, JMM:08} where the direction is referred to as alignment and the global direction is not predefined. Direction can also be applied as an overall requirement of the swarm as discussed in~\cite{HAY:08, IGMFM:08} and can be applied from the readings from a sensor. If the swarms is to be used in an open air environment covering a large demographic the tendency is to utilise a GPS sensor.

\section{Computational swarms}
Although the basis of swarming has been derived from nature and the benefits realised through transferring those concepts into the robotics domain there is another sub domain of robotic swarming. When modelling an environment the high level concept of swarming can be used by taking the concept of multiple agents (multi-agent) and then centrally managing those agents to perform a task~\cite{MGFND:05}. This gives rise to a formation based swarm whereby there is a model of known environment with full access to all the positional information, the model then calculates positional requirements for all the agents. This is different to natural swarms that are predominantly based on localised proximity field effects which are the ranges used to determine where other nearby agents are located and are autonomous~\cite{BAF:06}. There is therefore limited autonomy within the agents, this concept of the centrally controlled coordination is not seen in nature.

\subsection{Centralised swarms}
Centralised swarms are controlled from a `master' coordinator. The agents themselves are autonomous in terms of their function and having their own subsystems but the agents positional coordination is relayed to it from a central source~\cite{SB:93, MYP:09}. This is different to swarms that utilise an internal communications infrastructure to negotiate roles and exchange information~\cite{OFM:07}. The algorithm to control the swarm is centrally executed and due to this there are factors that limit the size of swarm that can be controlled. 
The main factor that limits the size of the swarm is the time required for messages to be transferred too, and propagated throughout the swarm, secondly the central processing of the algorithm will be complex as it has to calculate multiple agent locations rather than a single movement. The processing issue can be overcome by increasing the performance of the central controller but this will not effect the communications problem. 
This type of swarm works well when creating predefined structures such as the tower building swarm~\cite{EG:11} or the knot tying quadcopters~\cite{AZMD:15} that have been developed as part of the research projects ran by Raffaello D'Andrea in Dynamic Systems and Control at ETH Zurich. In 2016 D'Andrea received the IEEE award for Robotics and Automation~\cite{I:15}. 
The centralised controller can relay messages in different ways such as utilising an inter-agent communications architecture where the agents communicate with each other in a mesh network, this has an inherent $n^2$ messaging propagation issue. The communications infrastructure can be an independent platform outside of the swarm. Using an independent infrastructure does allow a wider distribution of a swarm~\cite{NVC:15} but still increases the response time of the agents to move to a new position, again this delay is due to message propagation.  

\section{Swarming applications}\label{sec:SwarmApplications} 
Many industries require the exploration or reconnaissance of environments that are not easily accessible by a human. If we consider disaster areas following earthquakes, or environments that are simply hard to survey due to their size, such as large scale commercial farming~\cite{BSB:15, HGCTREA:15}. There is also the exploration of underground environments or reconnaissance~\cite{IBE:08} of sub terrain or enclosed spaces. In mining environments, for example, the environment can be a labyrinth of tunnels that are extremely dangerous due to rock falls, toxic gas etc. these environments can also consist of many rejoining routes and dead ends. Exploration of these environments is, on occasion, necessary and invariably this type of work is best performed by some form of drone or autonomous vehicle based mechanism. There needs to be a `tool' to facilitate this type of work, swarms of autonomous robots (aerial and land based) are fast becoming that tool~\cite{SOM:12, LG:14}. 

An example of a large implementation of a swarming platform is Project Loon~\cite{KS:ND, G:13, RAV:13}. Google have completed trials and are now creating aerial platforms with high altitude balloons to provide communication infrastructures in remote areas of New Zealand~\cite{CBS:13, NZH:13}. There are also smaller scale projects investigating the use of swarms in surveying crops to check the health of plants~\cite{BSB:15}, so that remedial action can be taken to improve yields. The forestry commission have carried out surveys of forest environments. All these applications require the agents to not only coordinate themselves within the swarm environment but the agents must also carrying sensor arrays to detect the environmental conditions.

There is a view that swarms can be made to interact with humans and trials have taken place that use swarms to assist individuals in the fire service~\cite{PAWN:11} as part of the GUARDIAN project~\cite{SALGVPJ:08}. 

This thesis will be based upon potential reconnaissance of dangerous or inaccessible environments that have an unmapped environment, if we consider a disaster event such as a building collapse~\cite{STO:05} then the environment will be unknown. These tasks could be carried out several ways, a single autonomous explorer type robot could be used similar to the way Chernobyl was investigated~\cite{JA:98}. An individual agent could plot a path through the environment using a maze solving algorithm but the trend now is for swarms to be used.  

Given a swarm consists of many agents, there is a potential for redundancy, this redundancy is advantageous in that there is no single point of failure. Another feature of swarms is that, depending upon the application, it is possible to use small low cost agents~\cite{STO:05} and therefore the application of the swarm can be more cost effective.

Research has shown that the use of swarms has many benefits including: 
\begin{itemize} 
\item The speed at which an area can be traversed based upon the algorithm employed.
\item Recovery of the swarm from a agents failure to continue with a given task.
\end{itemize} 

Riano~\cite{RM:11} has also shown that there are hidden benefits in using swarms due to the emergent behaviours of group dynamics which can assist reconnaissance, one such feature is swarm reconstitution as it migrates around obstacles. This thesis will propose a mechanism to increase the ability of swarm to utilise this behaviour, specifically by the reduction of swarm voids~(\autoref{sec:ConcaveVoidReduction1}), this will build upon the work by Geunho Lee and Nak Young Chong~\cite{GN:08}.

Another way of employing a swarm is to exploit an emergent behaviour to `swamp' an area such that it fills or `floods' the entire area to give 100\% coverage of the designated space~(\autoref{chapter:flooding}), this is achieved by increasing the cohesion and repulsion fields until the resultant vectors can no longer settle to a background level and the internal `forces' increase (magnitude), this flooding feature does not require a GPS directional bias and is based upon a swarm interacting with its environment~\cite{BAFVM:06}, this is a area of research in swarming where there is limited research.

The swarms of interest in this thesis will be able to adapt to an environment in terms of their ability to disperse their agents and reconstitute the swarm structure if an opportunity arises, this will be in the case of a swarm passing obstacle on a designated path and as a stationary swarm surrounding an object.

\section{Focus of the thesis}
Research into swarming algorithms is carried out using both physical implementations~\cite{DOR:09, DTGT:04} and also through software simulations~\cite{BAF:06, GP:02}. All research evaluation in this thesis will be carried out in the form of software simulations to determine the effect of an algorithm and to simulate various environmental scenarios.

This thesis will focus on Boid based swarms and will incorporate, where necessary, the directional component as an overall requirement of the swarm as opposed to using alignment. The thesis will also look at emergent behaviours in swarms both with and without a directional bias. The application of the global direction will be carried out in different ways using different coordinator selection algorithms. 

\section{Contributions}
This thesis proposes to research and develop algorithms to allow a swarm of autonomous vehicles to cooperate as a swarm and for that swarm to be adaptable in terms of how its goal is obtained. The thesis will also be identifying how to analyse the resultant characteristics an algorithm creates and a means to compare the effects. The autonomy of the agents in this thesis will be to allow the agents to be aware of each other at a local rather than global level and to allow the swarm's infrastructure to develop specialisations in functionality when necessary. These constraints will allow the swarm to maintain its structure and therefore carry out specific functions as required in an appropriate manner. In some cases this may also reduce the gross energy consumption of the swarm. 

The nature of the swarm is to be considered as a collection of autonomous vehicles which can be adapted to carry out functions such as search and discovery over large areas, or surveillance (data collection) in an inhospitable environment. This is similar to the work in the SmartBot project that is looking to get autonomous agents to collaborate~\cite{DOR:09, DTGT:04, MFGAB:03}. The swarming component of this thesis will extend that research and focus on swarm coordination and analysis of the behaviours the algorithms generate.

This thesis proposes to take its lead from the natural world. This approach is not unique as can be seem from the development of the Bee Algorithm~\cite{PC:09} for swarm modelling~\citep{HER:11} and also multi-agent environments (a swarm is one such environment). The trend within these projects is to mimic the natural world by implementing mathematical models~\cite{GK:07} that reflect phenomena found in nature~\cite{MYP:09}. 

One of the initial directions of research into swarming looked at the need for agents in a swarm to be able to communicate with one another~\cite{ABN:93, FAP:05, KD:99, MOR:05}, initial research in this area looked at maintaining the swarms form as a whole and for inter-agent communications. There is an opinion that the agents could have a central communications infrastructure that is independent of the agents~\cite{NVC:15}, however there is also a view that communications could limit or impair a swarms functionality~\cite{HTM:09, IGMFM:08}. In some cases the research has shown that localising communications to just neighbours would be advantageous~\cite{HSWN:10}. The introduction of communications has allowed for concurrency and for swarms to become stateful i.e. at a given point in time all agents have data pertaining to the rest of the swarm~\cite{MFGAB:03}. This research has developed into cooperative control within a group of autonomous vehicles and coordinating their efforts such that a searching algorithm can be implemented to cover a given area; however there is a need for a `coordinator' to store search maps which increases the resource requirements of a swarm's agents~\citep{PYP:01}.

One major aspect of this thesis is to attempt to eliminate the need for a communications protocol, however, the agents will utilise a local sensing infrastructure for neighbour detection. This thesis proposes to look into the optimisation of power requirements for a swarm and also consider optimisation of swarm resources.

The thesis proposes to show that based upon the algorithms being developed they can be analysed to show which algorithms are developing the most efficient use of their motion and under which circumstances that wasted motion can be reduced.

The thesis proposes to show that emergent behaviours can be utilised to provide a more predictable swarm structure and increase the effectiveness of some swarm functions such as obstacle avoidance and area coverage.
 
\section{Aims}
This thesis aims to identify and develop a techniques to determine the effect an algorithm has upon a swarm, this will be in the form of a metric that can be applied to multiple algorithms to investigate their effects. The thesis will also look at techniques to eliminate the need for a communications infrastructure within a swarm while maintaining a swarms fitness for purpose. There will also be a focus on identifying and realising emergent behaviours that could be utilised in specific situations. When looking at how a swarm model can be applied to specific tasks that require area coverage a mechanism will be investigated that will promote void reduction in a swarms structure. there will also be a focus on how to provide area flooding with a swarm through modifications to the field effects that govern a swarms structure.

\section{Structure of the thesis}
The rest of the thesis will be structured as follows: Chapter 2 will cover methods tools and techniques used in defining a swarm. This will include how the techniques used to create different swarming structures and how the swarms analysed in this thesis have been created. This will include an overview of the simulator develop as part of the thesis. Chapter 3 will discuss the development and application of a metric to analyse the impact of algorithms on swarm internal movement and how this metric can be used to compare different swarming algorithms effects upon a swarm. Chapter 4 will discuss methods of coordinating goal based swarms and identify the changes these algorithms have upon the system by using the metric developed in Chapter 3. Chapter 5 will examine the emergent behaviours of swarms and how by adjusting some of the basic coordination algorithms it is possible to produce a swarming effect that can be applied to specific requirements. Chapter 6 will then bring together all the findings from the thesis.